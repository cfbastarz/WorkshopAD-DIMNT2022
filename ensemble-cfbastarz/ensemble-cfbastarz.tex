% Instituto Nacional de Pesquisas Espaciais - INPE
% https://github.com/cfbastarz/EstiloBeamerINPE
% V1.2
% carlos.bastarz@inpe.br (15/04/2021)

\documentclass[10pt,aspectratio=169]{beamer}

% Tema padrão (base)
\usetheme{default}

\include{./meus_pacotes}
\pgfplotsset{width=4.5cm,compat=1.8,every tick label/.append style={font=\tiny}}
\pgfmathdeclarefunction{gauss}{2}{%
\pgfmathparse{1/(#2*sqrt(2*pi))*exp(-((x-#1)^2)/(2*#2^2))}%
}

\newcommand{\unilogo}{
  \setlength{\TPHorizModule}{1pt}
  \setlength{\TPVertModule}{1pt}
   % textblock{}{x,y}: pos(x) = leftUpperCorner + (x * \TPHorizModule), pos(y) = leftUpperCorner - (y * \TPVertModule)
  \begin{textblock}{1}(25,2)
    \includegraphics[width=45pt,height=45pt]{Logomarca2020.png}
  \end{textblock}
  } 

%%%%%%%%%%%%%%%%

\tikzstyle{process1} = [rectangle, rounded corners, minimum width=3cm, minimum height=1cm, text centered, draw=black, fill=MaterialBlue!30]

\tikzstyle{process2} = [rectangle, rounded corners, minimum width=3cm, minimum height=1cm, text centered, draw=black, fill=MaterialYellow!30]

\tikzstyle{process3} = [rectangle, rounded corners, minimum width=3cm, minimum height=1cm, text centered, draw=black, fill=MaterialGrey!30]

\tikzstyle{process4} = [rectangle, rounded corners, minimum width=3cm, minimum height=1cm, text centered, draw=black, fill=MaterialOrange!30]

\tikzstyle{process5} = [rectangle, rounded corners, minimum width=3cm, minimum height=1cm, text centered, draw=black, fill=MaterialGreen!30]

\tikzstyle{process6} = [rectangle, rounded corners, minimum width=3cm, minimum height=1cm, text centered, draw=black, fill=MaterialRed!30]

\tikzstyle{process7} = [rectangle, rounded corners, minimum width=3cm, minimum height=1cm, text centered, draw=black, fill=MaterialPurple!30]

\tikzstyle{process8} = [rectangle, rounded corners, minimum width=3cm, minimum height=1cm, text centered, draw=black, fill=MaterialCyan!30]

\tikzstyle{arrow} = [thick,->,>=stealth]      

\setbeamerfont{framesubtitle}{size=\large}

\include{./meu_tema}

% Informações da capa da apresentação
\title{Assimilação de dados Híbrida visando a Previsão por Conjunto}
\author{Carlos Frederico Bastarz\\ \href{https://github.com/cfbastarz}{\faGithub} \href{http://lattes.cnpq.br/2410960909883784}{\faGraduationCap} \href{https://www.researchgate.net/profile/Carlos_Bastarz}{\faResearchgate} \href{mailto:carlos.bastarz@inpe.br}{\faEnvelope}}
\institute{\textbf{\small{Workshop DIMNT}\\\vspace{0.5em}\footnotesize{``A Assimilação de Dados nas Componentes do Sistema Terrestre:\\Status e Perspectivas Futuras no Contexto do MONAN''}}}
\date{
	 06 de outubro de 2022
}

\makeatletter
\@addtoreset{subfigure}{framenumber}% subfigure counter resets every frame
\makeatother

% A partir daqui inicia-se o documento
\begin{document}

% Capa (NÃO MODIFICAR)
{
\setbeamertemplate{footline}{} 
\begin{frame}
  \vspace{1cm}
  \titlepage
\end{frame}
}
 
% Reinicia do contador dos frames 
\addtocounter{framenumber}{-1}
 
% Sumário
\begin{frame}
\frametitle{Sumário}
\framesubtitle{\faListOl}
  \large\tableofcontents    
\end{frame}
 
\section{Introdução}

\subsection{Apresentação}

\begin{frame}{Introdução}
\framesubtitle{Apresentação}
  \begin{block}{Sistema de Previsão por Conjuntos Global:}
    \begin{itemize}
	 	  \item Inicialmente, estabelecido por \citeonline{coutinho/1999}, sob orientação do Dr. José Paulo Bonatti;
		  \pause
		  \item Método de perturbação da condição inicial atmosférica do MCGA baseado na análise do NCEP, utilizando EOFs \cite{zhangekrishnamurti/1999};
		  \pause
		  \item Recebeu contribuições a aprimoramentos de diversos colaboradores do CPTEC (Dr. Marcos Mendonça, Dra. Renata Weissmann, Dr. Christopher Cunningham e vários bolsistas e alunos da PGMET);
		  \pause
		  \item Executado nas máquinas NEC SX6, Cray XE6, portado para o Cray XC50 e testes iniciais na Egeon;
		  \pause
		  \item Resolução: TQ0126L028 (\textasciitilde100 km, 28 níveis sigma), com 15 membros, com horizonte de previsões de 15 dias.
	  \end{itemize}
  \end{block}
\end{frame}

\subsection{Método de Perturbação MB09}

\begin{frame}{Planos Futuros}
\framesubtitle{Método de Perturbação MB09}
  \vspace{-1em}
  \begin{figure}[t]
    \centering
    \includestandalone[width=0.75\linewidth]{./diagrama3}
  \end{figure}
\end{frame}

\section{Estado Atual}

\subsection{Operação}

\begin{frame}{Estado Atual}
\framesubtitle{Operação}
  \begin{block}{Situação Operacional:}
    \begin{itemize}
      \item Em 2021 foi concluída a transição da suíte oensMB09 para a máquina XC50;
    	\pause
			\item Na época, a versão do modelo atmosférico usado pelo oensMB09, não era operacional no XC50 - aproveitou-se a oportunidade para se fazer a atualização para a versão operacional do modelo BAM (em coordenada sigma);
			\pause
			\item A resolução foi mantida (TQ0126L028, \textasciitilde100 km e 28 níveis sigma) com 15 membros, produzidos a partir da análise atmosférica do NCEP;
			\pause
			\item Com a transição e o upgrade do modelo atmosférico, foi estabelecida a versão 2.2.0 \cite{figueroaetal/2016} do oensMB09 - \textbf{ainda não operacional};
			\begin{itemize}
			  \item Relatório de transição: resumido \href{https://s0.cptec.inpe.br/webcptec/sites/dmd/Avalia\%C3\%A7\%C3\%A3o-Modelo-Ensemble-Global-v1.1-2021.pdf}{(\faFile[regular] link)} e expandido \href{https://www.dropbox.com/s/je8q92jpwzc1bnz/16.\%20Relat\%C3\%B3rio\%20-\%20Bastarz\%20et\%20al.\%2C\%202021.pdf?dl=0}{(\faFile[regular] link)};
			\end{itemize}
			\item Aplicação do método de perturbação MB09 para a previsão subsazonal (TQ0126L042, 5 membros) - \textbf{em vias de operacionalização} \cite{guimaraes/2020, guimaraes/2021}.
    \end{itemize}
  \end{block}
\end{frame}

\subsection{Previsão de Precipitação - 10 dias}

\begin{frame}{Estado Atual}
\framesubtitle{Previsão de Precipitação - 10 dias}
  \vspace{-1em}
  \begin{figure}
    \centering
      \subfigure[CMORPH]{\includegraphics[width=0.375\textwidth]{./figs/cmorphjja.001_trim.jpeg}}
      \subfigure[BAM TQ0666L064]{\includegraphics[width=0.375\textwidth]{./figs/bamjjafct240.001_trim.jpeg}}\\\vspace{-0.5em}
      \pause
      \subfigure[ENS MEAN MCGA TQ0126L028]{\includegraphics[width=0.375\textwidth]{./figs/smcmeanjjaxe6fct240.001_trim.jpeg}}
      \pause
      \subfigure[ENS MEAN BAM TQ0126L028]{\includegraphics[width=0.375\textwidth]{./figs/smcmeanjjaxc50fct240.001_trim.jpeg}}
      \caption{}
  \end{figure}
\end{frame}

\subsection{Skill e Espalhamento}

\begin{frame}{Estado Atual}
\framesubtitle{Skill e Espalhamento}
  \begin{figure}[H]
    \centering
      \subfigure[Correlação de Anomalia]{\includegraphics[width=0.48\textwidth]{./figs/Untitled.005_trim.jpeg}}
      \pause
      \subfigure[Espalhamento X REQM]{\includegraphics[width=0.48\textwidth]{./figs/Untitled.007_trim.jpeg}}
      \caption{}
  \end{figure}
\end{frame}

\section{Planos Futuros}

\subsection{Assimilação 3DEnVar}

\begin{frame}{Planos Futuros}
\framesubtitle{Assimilação 3DEnVar}
  \vspace{-0.75em}
  \begin{figure}[t]
    \centering
    \includestandalone[width=0.75\linewidth]{./diagrama1}
  \end{figure}
\end{frame}

\subsection{Assimilação 3DEnVar+MB09}

\begin{frame}{Planos Futuros}
\framesubtitle{Assimilação 3DEnVar+MB09}
  \vspace{-0.75em}
  \begin{figure}[t]
    \centering
    \includestandalone[width=0.75\linewidth]{./diagrama2}
  \end{figure}
\end{frame}

\begin{frame}{Assimilação 3DEnVar+MB09}
\framesubtitle{Um Monte de Membros}
  \begin{block}{Juntando tudo:}
    \begin{itemize}
  	  \item Combinação entre as técnicas EnKF e MB09;
  		\pause
  		\item Vantagem: possibilidade de usar a mesma análise atmosférica para produzir o ensemble usando o framework da assimilação (i.e., 3DEnVar);
  		\pause
  		\item Este exercício servirá para testar as técnicas para serem aplicadas ao MONAN;
  		\pause
  		\item O ensemble do 3DEnVar (laranja) parece ser diferente do ensemble gerado pelo MB09 (rosa e azul).
  	\end{itemize}
  \end{block}
  \pause
  \begin{block}{Então...}
  	\begin{itemize}
  		\item O EnKF fornece características que complementam aquelas geradas pela e EOF? 
  		\item A análise determinística do 3DEnVar é um bom substituto para a análise do NCEP?
  	\end{itemize}
  \end{block}
\end{frame}

\begin{frame}{Assimilação 3DEnVar+MB09}
\framesubtitle{Um Monte de Membros}
  \begin{columns}[t]
    \begin{column}{.5\textwidth}
   	  \vspace{-1.5em}
    	\begin{itemize}
        \item Ponto sobre São Paulo (46S;23W);
        \item Previsão da PSNM 15 dias;
        \item Válido para 2012122000-2013010400.
    	\end{itemize}
    	\vspace{1em}
    	\begin{itemize}
			  \item O EnKF fornece características que complementam aquelas geradas pela e EOF? 
				\item A análise determinística do 3DEnVar é um bom substituto para a análise do NCEP?
				\item Contribuições da meteorologista Mirlen Filgueira (PCI/INPE, entre nov./2021 e abril/2022).
			\end{itemize}
    \end{column}
    \begin{column}{.5\textwidth}
      \vspace{-2em}
			\begin{figure}[t]
			  \centering
			  \includegraphics[width=1.\linewidth]{./figs/spaguete71.png}
			\end{figure}
    \end{column}
  \end{columns}
\end{frame}

\begin{frame}{Assimilação 3DEnVar+MB09}
\framesubtitle{Análises 3DEnVar e NCEP para o oensMB09}
  \begin{columns}[t]
	  \vspace{2em}
    \begin{column}{.45\textwidth}
      \centering
    	Prevsisão de prec. a partir das análises do NCEP e 3DEnVar
      \begin{figure}[t]
			  \centering
				\includegraphics[width=0.8\linewidth]{./figs/goes13.jpg}
				\caption{GOES 13 IR (11 Jan 2013, 12Z)}
			\end{figure}
    \end{column}
    \begin{column}{.5\textwidth}
      \begin{figure}[H]
        \vspace{-2em}
				\centering
        \subfigure[CTR NCEP]{\includegraphics[width=0.45\textwidth]{./figs/prec_ctr_exp1_trim.jpg}}
        \subfigure[MEAN NCEP]{\includegraphics[width=0.45\textwidth]{./figs/prec_mean_exp1_trim.jpg}}\\
        \pause
        \subfigure[CTR 3DEnVar]{\includegraphics[width=0.45\textwidth]{./figs/prec_ctr_exp3_trim.jpg}}
        \pause
        \subfigure[MEAN 3DEnVar]{\includegraphics[width=0.45\textwidth]{./figs/prec_mean_exp3_trim.jpg}}
			\end{figure}
    \end{column}
  \end{columns}
\end{frame}

\section{Dificuldades e Desafios}

\subsection{Algumas Questões}

\begin{frame}{Dificuldades e Desafios}
\framesubtitle{Algumas Questões \faQuestionCircle}
  \begin{itemize}
	  \item Controlar o espalhamento do ensemble das análises e previsões;
		\pause
		\item Lidar com o custo computacional elevado:
		\begin{itemize}
		  \item Inteligência artificial é um caminho: uma rede neural pode emular o EnKF no 3DEnVar?
		\end{itemize}
		\pause
		\item Finalidade do ensemble (aplicação): ter maior resolução ou ter conjunto maior?
		\item Com um centro do nosso tamanho, com as dificuldades que temos, precisamos enxugar as nossas suítes:
		\begin{itemize}
			\item Assimilação 3DEnVar: assimilação + PNT det. + PNT por conjunto (tempo estendido e subsazonal), utilizando a mesma versão do modelo (com configurações adequadas para cada aplicação).
		\end{itemize}
		\pause
		\item É necessário ter mais pessoas envolvidas com essa atividade;
		\pause
		\item Temos que discutir a demanda por esse produto (previsão de tempo até 15 dias):
		\begin{itemize}
			\item Neste momento, a previsão subsazonal pode suprir essa demanda?
		\end{itemize}
		\pause
	\end{itemize}
\end{frame}

\begin{frame}
\frametitle{Referências Bibliográficas}
\framesubtitle{\faBookOpen~\faNewspaper[regular]~\faIcon[regular]{file}}
  \vspace{-1em}
  \footnotesize\bibliography{referencias}
\end{frame}

% Frame Final (NÃO MODIFICAR)
\usebackgroundtemplate%
{%
  \includegraphics[width=\paperwidth,height=\paperheight]{fundo_slide_inpe_sem_logo.png}%	
}

\begingroup
\setbeamertemplate{footline}{}
{\nologo
\begin{frame}
  \begin{figure}[H]
    \vspace{-4em}
		\centering
    \hspace*{1.5em}\includegraphics[width=1.\textwidth]{DesinacaoNominativaCentralizada2020.pdf}
	\end{figure}
\end{frame}
}
\endgroup

\end{document}
